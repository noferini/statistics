%%%%%%%%%%%%%%%%%%%%%%%%%%%%%%%%%%%%%%%%
%                                      %
% Athanassios Protopapas, October 2005 %
% Mini-example for using apa.cls       %
%                                      %
%%%%%%%%%%%%%%%%%%%%%%%%%%%%%%%%%%%%%%%%

\documentclass{cimento}
\usepackage{graphicx}  % got figures? uncomment this
%\bibliographystyle{apacite}
\usepackage{lineno}  % got figures? uncomment this
\usepackage{braket}
\linenumbers
\title{Statistical classification using different approaches based on the Bayes Theorem and the Quantum Mechanics formalism}

\author{F.~Noferini\from{ins:centrofermi}\from{ins:infn}}
\instlist{
  \inst{ins:centrofermi} Museo Storico della Fisica e Centro Studi e Ricerche ``Enrico Fermi'' - Rome, Italy
  \inst{ins:infn} INFN, Sezione Bologna - Bologna, Italy
  }
  
 %%The correct list of PACS numbers and definitions is available at www.aip.org/pacs/pacs2010/about.html
\PACSes{\PACSit{--.--}{\dots}
\PACSit{--.--}{\ldots}}

%\acknowledgements{Written at the request of the Prac\TeX\ journal editors. Comments may be sent to the author at protopap@ilsp.gr.}

%\shorttitle{APA style manuscript}
%\rightheader{APA style manuscript}
%\leftheader{A.\ Protopapas}

%\newcommand{\jpsi}{\ensuremath{\rm J/\Psi}}

\begin{document}

\maketitle                            
\begin{abstract}
In this work a novel statistical approach is presented.
The aim of such an appraoch is to classify events associating a probability to belong to different classes in a way
which allows to classify simultaneoulsy events both in a statistical way and on event-by-event basis.
In particular two different startegies were derived in a similar framework: one using the Bayes' theorem properties
and one using the Quantium Mechanics formalism which treats, at statistical level, the usual finite detector resolution as a quantum
wave function effect.
\end{abstract}

%\tableofcontents{}

\section*{Introduction}

The classification operation is a very general concept which may assume different meaning depending on the context in which
it is applied.
Here we are referring to the capability to discriminate different event classes on the basis of some
measured variables.
Because of a usual finite resolution in a single measurement different classes may be not completelly separable
in the sense that for a single event it is not possible to define with absolute certainty if it belongs to a class
or to another one.
In the case we are interested to perform measurement by selecting a specific event class there are usual two kinds of
approaches which can be exploited: a statistical approach which allows to unfold the finite resolution/spread of our
measurements to extract the number of event belonging to that class (for instance by fitting the distributions in the
variables we are able to measured); or defining some cuts on the variables which minimize the contamination of other classes.
In the first case we are able to extract the total number of events without, in principle, any efficiency loss and
contamination but we are not able to classify on event-by-event basis.
In the latter one we can associate each event to a class (useful to characterize the properties of the class in terms of
other variables/osservables) but we have to deal with corrections for efficiency loss and contamination.
The idea to associate a probability to each event is to extend the properties of the statistical approach, no
efficiency loss and no contamination by other classes, to the event-by-event approach by replacing the concept
of cut with the use of a  weight.
In this approach the probability used as a weight should guarantee that the effect of the contaminations by other
event classes goes to zero whena averaging on all the events.

In the first section we will present a general formalism to define probabilities under different assumptions
and to derive the properties in each scheme we presented.
In the second section we will focus on the two particular cases introduced before, Bayesian probabilities and Quantum probabilities,
extending the study to the multi-variable and multiple-classification scenarios.
In the third section some example will be reported to show the power and the limitations of the methods.
Finally, we will discuss our results reporting our conclusions and final remarks.

\section{Connection between statistics and event-by-event appraoch}
\subsection{The foundamental relation}
We want to consider the case with a sample of data composed by $N_{type}$ classes distinuguished via the measurement of signal $S$ measured by a specific detector.
Assuming to know exactly what is the detector response for each event class expressed by $p_{i}(S)$ with i=1,2,...,$N_{type}$ and $\int p_{i}(S) dS = 1$.

A priori we don't know what the abundances, $M_{i}$, are for the different
event classes. We just know the total event multiplicity $M =
\sum\limits_{i}^{N_{type}} M_{i}$, while $M_i$ is what we usually want to measure.
Keeping them as a variable we can define the following relation for a generic function $f(S)$ of the signal:

\begin{equation}
\label{Sec1:BasicRelation}
\sum\limits_{j}^{M} f(S_j) = \sum\limits_{i}^{N_{type}} M_{i} \int p_{i}(S) f(S) dS,
\end{equation}
such a relation is valid because on average the signal distribution for a
given class has to be distributed accordingly to the know detector response
for each event class separatelly. Here  $S_j$ refers to the signal
measured in the $j$-th event.

It is worth to note that the relation in eq.~\ref{Sec1:BasicRelation} is valid
for any function $f(S)$ depending on the signal and
it is interesting to see what happens when we replace it with some specific functions.

\subsection{Probability from the Bayes theorem}
First of all we are going to introduce a probability defined using the Bayes theorem. Such a probability, which depends on the measured signal, was particulary used in the ALICE experiment (reference) to identify particles (in that case event classes are represented by different particles type) and therefore it was reported as a good candidate to define the probability we are looking for.

Using the Bayes theorem where $p_{i}$ represent the conditional probability (the probability that an event of a class, $H_{i}$, relases the measured signal) $P(S|H_{i})$ it is possible to derive the posterior probability (the probability that the measured signal belongs to an event of the class $H_{i}$) $P(H_{i}|S)$ via the formula (for simplicity we will refer to it simply with $P_{i}(S)$):
\begin{equation}
\label{Sec1:BayesianProb}
P_{i}(S) = P(H_{i}|S) = \frac{C_{i} P(S|H_{i})}{\sum\limits_{i}^{N_{type}}C_{i} P(S|H_{i})} = \frac{C_{i} p_{i}(S)}{\sum\limits_{i}^{N_{type}}C_{i} p_{i}(S)},
\end{equation}
where $C_{i}$ is usually called the prior probability which is the best guess of the abundances for all the event classes, expected to be as closer as possible to the the real one $M_{i}$. The main property of the Bayesian probability is that by definition $\sum\limits_{i}^{N_{type}} P_{i}(S) = 1$.

The use of Bayesian probability is applied by replacing in eq.~\ref{Sec1:BasicRelation} $f(S)$ with $P_{i}(S)$, i.e. by counting all the events with the probability as a weight. The way the Bayesian probability is defined guarantees very important properties:
\begin{eqnarray}
\label{Sec1:BayesianProp}
\sum\limits_{j=1}^{M}P_{i}(S_j) = M_{i},{\rm\ if\ {\it C_{i} = M_{i}}}, \\
\label{Sec1:BayesianProp2}
C_{i} \rightarrow M_{i} {\rm\ via\ iterative\ procedure}.
\end{eqnarray}
While the property in eq.~\ref{Sec1:BayesianProp} can be easily demonstrated by replacing in eq.~\ref{Sec1:BayesianProb}  $C_{i}$ with $M_{i}$, the latter one requires more extended computation.
In particular for $C_{i} \neq M_{i}$ eq.~\ref{Sec1:BayesianProp} can be rewritten (after some calculations) in the form:
\begin{eqnarray}
\label{Sec1:BayesFullCalc}
\sum\limits_{j=1}^{M}P_{i}(S_j) = M_{i} \int \frac{1}{1+\frac{\sum\limits_{k \neq i} \left( \frac{C_k}{C_i} - \frac{M_k}{M_i} \right) p_{k}(S)}{1 + \sum\limits_{k \neq i} \frac{M_k}{M_i} p_k(S)}} p_{i}(S)dS \sim \\
\sim M_{i} \int  \left( 1+\frac{\sum\limits_{k \neq i} \left( \frac{M_k}{M_i} - \frac{C_k}{C_i} \right) p_{k}(S)}{1 + \sum\limits_{k \neq i} \frac{M_k}{M_i} p_k(S)} \right) p_{i}(S)dS = \\
= M_{i}  \left( 1 + \sum\limits_{k \neq i}  \left( \frac{M_k}{M_i} - \frac{C_k}{C_i} \right) \int  \frac{p_{k}(S)p_{i}(S)dS}{1+\sum\limits_{l \neq i} \frac{M_l}{M_i} p_l(S)} \right).
\end{eqnarray}
Providing that $\frac{C_k}{C_i} - \frac{M_k}{M_i}$ becomes smaller and smaller via iterative procedure and that $\int p_{k}(S)p_{i}(S)dS \ll 1$ for $k \neq i$ (good separation), then $C_i \rightarrow M_i$

For the case reported by ALICE (reference) it was shown that in less than 10 iterations the convergence $C_{i} \rightarrow M_{i}$ was observed.
The fact that an iterative procedure is needed depends on the presence in the
Bayesian probability expression of a term, the prior probability, which has to
be extracted from the data sample. The speed of the converegence depends on
the separation of the event classes with respect the signal $S$ and it starts
to be quite slow when it is worse than $1\sigma$ (if assuming Gaussian distributions).
Moreover, it should be noted that we are assuming that the detector responses and the prior probabilities are constant in the whole data sample.
For the case they depend on other variables (like $p_{T}$ in the case of the ALICE example) the sample has to be divided in subsample in order to guarantee that within each subsample the parameter assumed are constant. Because of the finite statistics of the data sample such an operation is not always possible leading to possible systematic effects.

\subsection{Probability and Quantum Mechanics formalism}
For the reasons exposed in the previous paragraph we looked for a different
definition of the probability to be independent of the assumption on the prior
probabilities. In this search we discovered that replacing the Bayesian
probability $P_{i}(S)$ with the detector response probability $p_{i}(S)$ 
we gain many useful properties.

Replacing in eq.~\ref{Sec1:BasicRelation} $f(S) \rightarrow p_{k}(S)$ the
following experession is obtained:
\begin{equation}
\label{Sec1:QSmain}
\sum\limits_{j}^{M}p_{k}(S_j) = \sum\limits_{i=1}^{N_{type}}M_{i}\int p_{k}(S)
p_{i}(S) dS = \sum\limits_{i=1}^{N_{type}}M_{i} a_{ki}, 
\end{equation}
where $a_{ki}=a_{ik}$ are quantities depending only on the detector responses
for the different types of signals and they can be computed {\it a priori} if responses are well known.
For instance in the case of Gaussian responses with constant resolution
$\sigma$ and different mean values ($p_{i}(S) =
exp(-(S-\bar{S}_{i})^{2}/2\sigma^{2})/\sqrt{2\pi\sigma^{2}}$) $a_{ki}$ can
be calulated analitically:
\begin{equation}
\label{Sec1:GaussianMatrixElements}
a_{ki} = a_{ik} = \frac{exp(-\Delta_{ik}^{2}/4)}{\sqrt{8\pi\sigma^2}},
\end{equation}
where $\Delta_{ik}=(\bar{S}_{i} - \bar{S}_{k})/\sigma$ represents the
separation, in number of $\sigma$s, between two different classes of events.
In the trivial case where the $\Delta_{ik}$ values are very large (for $i \neq
k$) $a_{ki} \rightarrow \delta_{ik} / \sqrt{8\pi\sigma^2}$.
In the case the detector responses are Gaussian but with different
widths the previous results are still valid by replacing $\sigma$ with
$\sigma_{ik}$,  being $\sigma_{ik}^2 = \frac{\sigma_{i}^2 + \sigma_{j}^2}{2}$.

Therefore, if the $a_{ik}$ are known eq.~\ref{Sec1:QSmain} leads to the
relation:
\begin{equation}
\label{Sec1:QSformula}
\vec{O} =
 \left(
\begin{array}{c}
o_1 = \sum\limits^{M} p_{1}\\
\ldots \\
o_{N_{type}}=\sum\limits^{M} p_{N_{type}}
\end{array}
\right) =
\left(
\begin{array}{ccc}
a_{1,1} & \ldots & a_{1,N_{type}} \\
\ldots & \ldots & \ldots \\
a_{N_{types},1} & \ldots & a_{N_{type},N_{type}}
\end{array}
\right)
\left(
\begin{array}{c}
M_{1}\\
\ldots \\
M_{N_{types}}
\end{array}
\right) =
\boldmath{A} \vec{M},
\end{equation}
where $\vec{O}$ represents the vector of observables which can be constructed from the data
sample, $\vec{M}$ the vector with the abundaces of the classes we want to
extract and $\boldmath{A}$ the matrix with $a_{ik}$ elements.

If $\boldmath{A}$ is known and can be inverted (all classes are
well separated as distributions in the variable $S$) the previous relation can
be used to extract $\vec{M}$ from $\vec{O}$:
\begin{equation}
\label{Sec1:QSeqInv}
\vec{M} = \boldmath{A^{-1}} \vec{O}
\end{equation}
This approach allows to extract the real abundances without the need of
an iterative procedure as in the Bayesian approach because the prior
probability are no longer included in the formulation.

This is not the full story. Even if eq.~\ref{Sec1:QSeqInv} is
obtained after having summed on all the events in our sample, it is possible
to define the same relation on a single event. This operation leads to some
analogies with the Quantum Mechanics formalism.

Let consider a status $\psi_{i}(S)$ which is the $i$ component of the vector
defined as:
\begin{equation}
\label{Sec1:psi}
\vec{\Psi}(S) = \boldmath{A^{-1}} 
\left(
\begin{array}{c}
p_{1}(S)\\
\ldots \\
p_{N_{types}}(S)
\end{array}
\right)
\end{equation}
The first property we can find, by contruction, is:
\begin{equation}
\label{Sec1:psiProp1}
\int \psi_{i}(S) p_{k}(S) dS = \delta_{ik},
\end{equation}
which immediately leads to the relation (by replacing eq.~\ref{Sec1:psiProp1} in eq.~\ref{Sec1:BasicRelation}):
\begin{equation}
\label{Sec1:psi}
\sum\limits_{j=1}^{M}\psi_{i}(S_j) = M_{i}
\end{equation}
Therefore, the inversion of the $\boldmath{A}$ matrix can be performed at the
level of a single event to define an amplitude for each event class (then a
vector of amplitudes) and the sum of all the events can be performed using
each amplitude as a weight.

If we compare $P_{i}(S)$, from the Bayesian approach, with $\psi_{i}(S)$ in
this novel approach, we can identify some important differences:
\begin{itemize}
\item $0 < P_{i}(S) < 1$ while $\psi_{i}(S)$ has no limitation (it can be also
  negative),\\
\item $\psi_{i}(S)$ doesn't depend on prior probabilities (particle
  abundances) but only on the detector responses,\\
\item On average the contribution to
  $\psi_{i}$ of the events belonging to a class $k$ is zero if $i \neq k$, and
  1 if (i = k) (orthogonality)
\end{itemize}

The use of Quantum Mechanics formalism may be pushed over.
We can define:
\begin{eqnarray}
\label{Sec1:QMdef1}
p_{i}(S) = \bra{i} \\
\psi_{i}(S) = \ket{i}
\end{eqnarray}
where the transformation from $\bra{bra}$ to $\ket{ket}$ is provided by eq.~\ref{Sec1:psi}.
Basically, we are introducing a mixing matrix ($\boldmath{A^{-1}}$) mapping $\bra{bra}$
to $\ket{ket}$ in order to obtain ``states'' which are orthogonal:
\begin{equation}
\label{Sec1:Orto}
\braket{i|k} = \int p_{i}(S) \psi_{k}(S) dS = \delta_{ik}
\end{equation}

We can also represent our data sample in terms of our signal ($S$)
distribution as:
\begin{equation}
\label{Sec1:DataSample}
D(S) = \sum\limits_{k=1}^{N_{type}} M_{k} p_{k}(S) = 
\sum\limits_{k=1}^{N_{type}}M_{k}\bra{k} = \bra{D},
\end{equation}
so that:
\begin{eqnarray}
\label{Sec1:Distribution}
\sum\limits_{j=1}^{M} f(S_j) = \int D(S) f(S) dS \\
\sum\limits_{j=1}^{M} \psi_{i}(S_j) = \int D(S) \psi_{i}(S) dS = \braket{D|i} = \sum\limits_{k=1}^{N_{type}}M_{k}\braket{k|i}=M_{i}
\end{eqnarray}
The sum over the data sample of the amplitude for a given class corresponds to
the projection of the distribution on the ``state'' (event class) we are
selecting. This means that the use of the amplitude as a weight when
constructing any observable allow to put to zero (statistically) the contribution of all the
events belonging to the wrong type and to preserve the contribution of the
events of the right type as fully efficient.


\section{Generalization to multi-variables cases}
\subsection{The multi-variables case}

\subsection{Events correlation: simultaneous classification of event pairs}

\subsection{Gaussian vs non-Gaussian response}



\section{Examples}
\subsection{Simple Gaussian case for two classes}

\subsection{Dependence on classes separation}

\subsection{Multi-class combination}



\section{Discussion}
\subsection{The effect of priors}

\subsection{Iterative vs analytical solutions}

\subsection{Advantages of the Bayesian approach}

\subsection{Advantages of the Quantum Mechanics formalism}



\section*{Conclusions}


%%%%%%%%%%%%%%%%%%%%%%%%%%%%     figure 5c  %%%%%%%%%%%%%%%%%%%%%%%%%
%\begin{figure}[!h]
%\begin{center}
%    \includegraphics[width=0.9\linewidth]{eps/plot_5.0_5.5.eps}
%\end{center}
%    \caption{$(N\sigma_{TPC},N\sigma_{TOF})$  distribution for all the species in 0-5\% PbPb collisions for $5.0 < p_{T} < 5.5~\rm{ GeV/}c$. The distributions were fitted assuming a Gaussian distribution plus exponential tail.}
%    \label{Fig:Nsigma2D3}
%\end{figure}
%%%%%%%%%%%%%%%%%%%%%%%%%%%%%%%%%%%%%%%%%%%%%%%%%%%%%%%%%%%%%%%%%%%%

\begin{thebibliography}{0}
%\bibitem{Adam:2015yta}  \BY{ALICE Collaboration (Adam, J. \etal)} \IN{Phys. Lett.}{B754}{2016}{360};

%\bibitem{} \BY{} \IN{}{}{}{};
%\bibitem{} \BY{} \IN{}{}{}{};
%\bibitem{} \BY{} \IN{}{}{}{};
%\bibitem{} \BY{} \IN{}{}{}{};
%\bibitem{} \BY{} \IN{}{}{}{};


%\bibitem{ref:apo} \BY{Einstein A. \atque Fermi E.}
%  \IN{Phys. Rev. A}{13}{1999}{12};
%  \SAME{69}{999}{1666}.
%\bibitem{ref:pul} \BY{Newton I.}
%  preprint INFN 8181.
%\bibitem{ref:bra} \BY{Bragg~B.}
%  \TITLE{Complete Works}, in \TITLE{Workers Playtime}, edited by \NAME{Tizio A. \atque Caio B.} (Unexeditor, Bologna) 1997, pp.~1-10.
\end{thebibliography}
\end{document}
